\pdfbookmark[1]{Abstract}{Abstract}
\begingroup
\let\cleardoublepage\relax
\let\cleardoublepage\relax

\chapter*{Abstract}

With the increase in computational power in the last years, neural networks have seen a grand revival that has brought along significant progress in many problems that were once considered too hard. However, this newfound power comes hand in hand with a raise in the difficulty of finding the optimal configuration for a given task.

In this work we propose an incremental evolutionary method for optimizing hyperparameters in neural networks. Starting with a feature selection phase, a population of neural networks is first evolved to find appropriate hidden layer configurations (\textit{structure optimization}); then, another population is evolved in order to find the best combination of learning rate, dropout rate and number of epochs (\textit{learning optimization}). Every one of these three steps can leverage implicit GPU parallelism and explicit CPU task distribution if the computational loads demand it.

Applied to classification of motor imagery tasks for brain-computer interfacing (\textit{BCI}), this method appears to produce very promising results in terms of accuracy, both using neural networks and some simpler classifiers like Support Vector Machines.\\

\textbf{Keywords:} Brain-computer interfaces (BCI) · Genetic algorithms · Artificial neural networks · Feature selection · Hyperparameter optimization


\vfill
