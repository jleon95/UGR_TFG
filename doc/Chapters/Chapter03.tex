\chapter{Motivation and objectives}\label{ch:objectives} 
%************************************************

\section{Motivation}

	In recent years, the field of machine learning has seen substantial advances. Due to the emergence of a wide array of novel techniques and the significant increase in computational power, state-of-the-art models outperform older methods in many applications. Deep Neural Networks are the most successful example, achieving results comparable to those of humans in tasks such as image recognition or game playing.

	Because of this, a certain degree of knowledge or familiarity with machine learning is a valuable asset for any Computer Science or Computer Engineering graduate. Thus, this work aims to serve as an opportunity to acquire more insight into the capabilities of neural networks in conjunction with related optimization methods. Additionally, since we will be using real data instead of typical testing datasets, it will provide some experience about tangible problems and their complexity.

	My personal motivation is not too far from the above. As a sci-fi reader, the idea of machines solving tasks better than humans is always present. True machine intelligence may not come soon, if it ever does, since there is a long road ahead. However, experience demonstrates that we are too early to either dismiss or take for granted, and so the only way to know is to try, step by step. Although not as fancy as what can be found in \textit{Hugo} and \textit{Nebula} award winners, machine learning is already a reality in terms of usefulness, and that is what I am interested in.

\section{Objectives}

	The objectives of this work are a natural consequence of everything stated up to this point. We can condense them as follows:

	\begin{itemize}

		\item
		Gain practical expertise on the use of neural networks (in particular, training parameters and optimal structures).

		\item
		Learn about different optimization techniques and algorithms that can complement neural networks. We are mainly looking at genetic algorithms.

		\item
		If possible, build a general enough codebase which can be reused and extended for further experiments.

	\end{itemize}