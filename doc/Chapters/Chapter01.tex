\chapter{Introduction}\label{ch:introduction}
%************************************************

\section{Brain-Computer Interfaces}

	In the famous movie \textit{The Matrix}, humans are plugged into the evil matrix through a port in the back of their skulls. In the 1995 Japanese animated movie \textit{Ghost in the Shell}, a number of characters also sport a similar device (this time voluntarily) in order to perform several tasks, a remarkable one being diving into virtual reality.

	These and many other variations of this sci-fi concept fall into the term \ac{BCI}. \acs{BCI}s function as communication channels that do not rely on peripheral nerves or muscles \cite{bcidef}; put in a practical way, they can allow for a compensation and even augmentation of some human abilities.

	While not as futuristic as in fiction, the prospects of current \acs{BCI} technology range from convenient to humanitarian---from brain-operated typewriters to medical diagnosis or advanced systems that aid paralyzed patients in their daily lives.

\section{Electroencephalography}

	As the name Brain-Computer Interface suggests, we need some sort of contraption in order to establish a link between our brain and a computer. Generally, it can be of one of three kinds:

	\begin{itemize}

		\item
		Invasive: these include the most powerful applications, like vision and mobility restoration; brain implants are an example of this. However, they come with important downsides, such as neurosurgery and its collateral effects.
		\item
		Non-invasive: when surgery is not appropriate, there exist other technologies that trade in precision for ease of use and lower operation costs. Among these we find \ac{EEG}.
		\item
		Partially invasive: a compromise between the two in the form of a not-so-aggressive procedure that only reaches the outside of the brain. \ac{ECoG} is a promising method.

	\end{itemize}

	In particular, the dataset employed in the experiments of this document was obtained with \acs{EEG}. This paradigm utilizes a set of electrodes placed on the scalp to record electrical activity from the brain; those readings are later useful for medical diagnosis or diverse engineering purposes (ours being one of them).

