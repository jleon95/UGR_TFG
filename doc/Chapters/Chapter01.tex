\chapter{Introduction}\label{ch:introduction}
%************************************************

\section{Brain-Computer Interfaces}

	In the famous movie \textit{The Matrix}, humans are plugged into the evil matrix through a port in the back of their skulls. In the 1995 Japanese animated movie \textit{Ghost in the Shell}, a number of characters also sport a similar device (although voluntarily) in order to perform several tasks, a remarkable one being diving into virtual reality.

	These and many other variations of this sci-fi concept fall into the term \ac{BCI}. \acs{BCI}s function as communication channels that do not rely on peripheral nerves or muscles \cite{bcidef}. Put in a practical way, they allow for a compensation and even augmentation of some human abilities.

	While not as futuristic as in fiction, the prospects of real \acs{BCI} technology range from convenient to humanitarian---from mind-controlled typewriters to advanced systems that aid paralyzed patients in their communication and daily lives.